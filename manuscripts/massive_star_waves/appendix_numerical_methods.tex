\section{Numerical Methods}
\label{app:numerical_methods}

We nondimensionalize Eqns.~\ref{eqn:continuity}-\ref{eqn:energy}.
Our nondimensional length scale $\Lnd$ is the radial coordinate of the outer boundary of the core convection zone.
We nondimensionalize the temperature as the temperature at the stellar core $\Tnd = T(r=0)$ and we use the central density to nondimensionalize the mass $\mnd = \rho(r=0)\Lnd^3$.
We use the central value of the nuclear generation rate $H_0 = (\rho \epsilon)(r=0)$ to define the nondimensional timescale $\tnd = (\mnd/[H_0\Lnd])^{1/3}$.
The important remaining control parameters are then the nondimensional viscosity $\nu_{\rm{nd}} = \nu \tnd/\Lnd^2$ and thermal diffusivity $\chi_{\rm{nd}} = \chi_R \tnd/\Lnd^2$.

Under this nondimensionalization, the equations of evolution become
\begin{align}
    &\grad\dot\vec{u} + \vec{u}\dot\grad\ln\rho_0 = 0
    \label{eqn:dedalus_nondim_anelastic}, \\
    \begin{split}
    &\partial_t \vec{u} + \grad\varpi + s_1(\grad T_0) + \vec{u} \mathcal{D} \\
        &\qquad- \nu_{\rm{nd}}\left(\grad\dot\bar{\vec{\sigma}} + \bar{\vec{\sigma}}\dot\grad\ln\rho_0\right)\\
    &\qquad= \vec{u}\times(\grad\times\vec{u}),
    \end{split}
    \label{eqn:dedalus_nondim_momentum}\\
    \begin{split}
        &\partial_t s_1 + \vec{u}\dot(\grad s_0) -\grad s_1 \dot \grad \chi_{\rm{nd}}  \\
        &\qquad- \chi_{\rm{nd}}(\grad^2 s_1 + [\grad s_1]\dot[\grad \ln \rho_0 + \grad\ln T_0]) \\
        &\qquad= - \vec{u}\dot\grad s_1 + \frac{\nu_{\rm{nd}}}{T_0}\mathcal{V} + \tilde{\mathcal{H}},
    \end{split}
    \label{eqn:dedalus_nondim_entropy}
\end{align}
where we define $\tilde{\mathcal{H}} = \mathcal{H}/(\rho_0 T_0)$ before nondimensionalizing.
The state variables are $\vec{u}$ the velocity, $s_1$ the fluctuations of specific entropy, and $\varpi$ the dynamic reduced pressure.
We define the stress tensor $\bar{\vec{\sigma}} = \Pi/(\rho_0\nu)$.
The background stellar structure is that of a $15 M_\odot$ MESA model (see appendices \ref{app:mesa_dedalus} \& \ref{app:mesa}).
This provides radial profiles of $\grad\ln\rho_0$, $\grad\ln T_0$, $\grad T_0$, $T_0$, $\grad s_0$, $\chi_R$, and the effective heating $\mathcal{H}$.
We refer the reader to appendix \ref{app:mesa_dedalus} for information on how these are calculated and what differences exist between MESA and Dedalus.

$\mathcal{D}$ is a radial profile of a damping coefficient.
For the runs with damping presented in sec.~\ref{sec:results_damping}, we set $\mathcal{D} = 0$ in the interior and $\mathcal{D} = f_b$ where $f_b$ is the frequency associated with the peak of $N^2$ in our simulation domain.
\todo{be more precise about $\mathcal{D}$}
%ncc_dict['sponge_S']['g'] = zero_to_one(r1S, r_inner + 2*L_shell/3, 0.1*L_shell)
%f_brunt = f['tau'][()]*np.sqrt(f['N2max_shell'][()])/(2*np.pi)
%ncc_dict['sponge_S']['g'] *= f_brunt


\todo{Update this to be true for this paper.}
We time-evolve equations \ref{eqn:dedalus_nondim_anelastic}-\ref{eqn:dedalus_nondim_entropy} using the Dedalus pseudospectral solver \citep[git commit 1339061]{burns_etal_2020} using timestepper SBDF2 \citep{wang_ruuth_2008} and safety factor 0.3.
All variables are spectral expansions of Chebyshev coefficients in the vertical ($z$) direction ($n_z = 512$ between $z=[0, 2.25]$ plus $n_z = 64$ between $z=[2.25, 3]$) and as ($n_x$, $n_y$) = (192, 192) Fourier coefficients in the horizontally periodic ($x$, $y$) directions.
Our domain spans $x \in [0, L_x]$, $y \in [0, L_y]$, and $z \in [0, L_z]$ with $L_x = L_y = 4$ and $L_z = 3$.
To avoid aliasing errors, we use the 3/2-dealiasing rule in all directions.
To start our simulations, we add random noise temperature perturbations with a magnitude of $10^{-6}$ to the initial temperature profile.

Spectral methods with finite coefficient expansions cannot capture true discontinuities.
To approximate discontinuous functions such as Eqns.~\ref{eqn:initial_T} \& \ref{eqn:initial_mu}, we define a smooth Heaviside step function centered at $z = z_0$,
\begin{equation}
H(z; z_0, d_w) = \frac{1}{2}\left(1 + \mathrm{erf}\left[\frac{z - z_0}{d_w}\right]\right).
\label{eqn:heaviside}
\end{equation}
where erf is the error function and we set $d_w = 0.05$.
The simulation in this work uses $\mP = 3.2 \times 10^3$, $\Ro^{-1} = 10$, $\rm{Pr} = \tau = 0.5$, $\tau_0 = 1.5 \times 10^{-3}$, and ${\kappa_{T,0} = \mP^{-1}[(\partial T/\partial z)_{\rm{rad}}|_{z=0}] / (\partial T/\partial z)_{\rm{rad}}}$

We produce figures \ref{fig:profiles} and \ref{fig:kippenhahn} using matplotlib \citep{hunter2007, mpl3.3.4}.
We produce figure \ref{fig:dynamics} using plotly \citep{plotly} and matplotlib.
All of the Python scripts used to run the simulations in this paper and to create the figures in this paper are publicly available in a git repository (\url{https://github.com/evanhanders/d3_stars}) and in a Zenodo repository \citep{supp}.
