\section{Transforms and Spectra}
\label{app:transforms}

All wave flux and poer spectra presented in this work are calculated from long time series of data of our state variables projected onto a 2D sphere.
Given a 3D data cube of a variable $a(t, \phi, \theta)$, we transform the data into frequency space as
\begin{equation}
a(t,\phi,\theta) \,\,\xrightarrow[\rm{SHT}]{}\,\,
A(t,\ell,m) \,\,\xrightarrow[\rm{FT}]{}\,\,
\hat{A}(f, \ell, m).
\end{equation}
To perform the spherical harmonic transform (SHT), we load the real grid-space data into a Dedalus \texttt{field} object, then convert the field to coefficient space.
Dedalus coefficient data must be divided by $\sqrt{2}$ at $\ell = 0$ data and $2$ for $\ell > 0$.
This returns two real values $b_1$ and $b_2$ for each ($\ell$, $m$) pair, and $A(t, \ell, m) = b_1 + \hat{j} b_2$ (where $\hat{j} = \sqrt{-1}$).
To perform the Fourier Transform (FT), we first measure the total number of temporal data points in the sample $N$ and retrieve the Hanning window $w$ of $N$ points with \texttt{numpy.hanning}.
We then use the \texttt{numpy.fft.fft} to transform the product of $w$ and $A(t, \ell, m)$, and normalize the output by $\sqrt{8/3}/N$ to retrieve $\hat{A}(f, \ell, m)$.
To obtain the frequencies, we use \texttt{numpy.fft.fftfreq} with $N$ data points and provide it the time step between each output in our data series.

To calculate a power-like quantity (power spectrum, wave flux spectrum), we multiply one transformed field with the complex conjugate of another, e.g., $P_A(f) = \hat{A}^* \hat{A}$.
To properly account for power in negative frequencies, we define $P_A(f) = P_A(f) + P_A(-f)$ for all $f \geq 0$.

To calculate the wave flux, we output $p = \varpi - 0.5 |\vec{u}|^2$ and $u_r = \vec{u}\dot\hat{e}_r$.
We take transforms per the above recipe and calculate the wave flux $W(f) = R^2\rho(R) \hat{P}^*\hat{U}_r$, where $R$ is the radial coordinate at which the shell is sampled.
