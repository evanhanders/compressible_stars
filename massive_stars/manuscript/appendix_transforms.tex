\section{Transforms and Spectra}
\label{app:transforms}

All wave flux and power spectra in this work are calculated from long time series of data of our state variables projected onto a 2D sphere.
Given a 3D data cube of a variable $a(t, \phi, \theta)$, we transform the data into frequency space as
\begin{equation}
    a(t,\phi,\theta) \,\,\xrightarrow[\rm{SHT}]{}\,\,
    \hat{a}_{\ell,m}(t) \,\,\xrightarrow[\rm{FT}]{}\,\,
    \hat{A}_{\ell,m}(f).
\end{equation}
We perform the spherical harmonic transform (SHT) using Dedalus.
Dedalus returns two coefficient amplitudes, $b_1$ and $b_2$, corresponding to the $\cos (m\phi)$ and $-\sin(m\phi)$ components.
From these, we construct the spherical harmonic amplitude $\hat{a}_{\ell,m}(t) = b_1 + i b_2$.

%Verify from documentation https://numpy.org/doc/stable/reference/routines.fft.html#module-numpy.fft
We calculate the Discrete Fourier Transform using NumPy and define its normalization such that
\begin{equation}
    \hat{A}_{\ell,m, f} = \frac{1}{N}\sqrt{\frac{8}{3}}\sum_{j=0}^{N-1} H_N(j)\hat{a}_{\ell,m}(t_j) \mathrm{exp}\left\{-2\pi i \frac{j f}{N}\right\}
\end{equation}
In this definition, $H_N(j)$ is the $j$th point of the Hanning window defined over $N$ total data points, and the factor of $\sqrt{8/3}$ accounts for this window.

To calculate a power-like quantity (power spectrum, wave flux spectrum), we multiply one transformed field with the complex conjugate of another, e.g., $P_A(f) = \hat{A}^* \hat{A}$.
To properly account for power in negative frequencies, we define $P_A(f) = P_A(f) + P_A(-f)$ for all $f \geq 0$.

To calculate the wave flux, we output $p = \varpi - 0.5 |\vec{u}|^2$ and $u_r = \vec{u}\dot\hat{e}_r$.
We take transforms per the above recipe and calculate the wave flux $W(f) = R^2\rho(R) \hat{P}^*\hat{U}_r$, where $R$ is the radial coordinate at which the shell is sampled.
\todo{call it wave luminosity and add 4$\pi$}%

%work on windowed FFTs to see convergence in time (of the peaks).
