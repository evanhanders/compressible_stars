\section{From MESA to Dedalus}
\label{app:mesa_dedalus}
We use a 15 $M_\odot$ ZAMS MESA stellar model to construct our background stratification.
From this model, we load the radial profiless of the mass $m$, radial coordinate $r$, density $\rho$, pressure $P$, temperature $T$, energy generation rate per unit mass $\epsilon$, opacity $\kappa$, logarithmic temperature gradlent $\nabla = d \ln T/d \ln P$, adiabatic temperature gradient $\gradad$, $\chi_{\rho} = (d\ln P /d\ln\rho)|_T$, $\chi_T = (d\ln P/d\ln T)|_{\rho}$, \brunt$\,$frequency $N^2$, specific heat at constant pressure $c_P$, luminosity $L_*$, convective luminosity fraction $f_L = L_{\rm{conv}}/L_*$, and sound speed $c_s$.
We secondarily calculate
\begin{align}
    &\frac{d\ln P}{dr} = -\frac{\rho g}{P},\\
    &\frac{d\ln \rho}{dr} = \frac{d\ln P}{dr}\frac{\chi_T}{\chi_\rho}\left(\gradad - \justgrad\right) - \frac{g}{c_s^2},\\
    &\frac{d\ln T}{dr} =  \frac{d\ln P}{dr}\justgrad,
\end{align}
and we define $L_{\rm{conv}} = L_* f_L$ and $dT/dr = T d\ln T/dr$ and gravitational acceleration $g = G m / r^2$ and specific entropy gradient $ds/dr = c_P N^2 / g$.
We calculate the radiative conductivity,
\begin{equation}
    k_{\rm{rad}} \equiv \frac{16 \sigma_{\rm{SB}} T^3}{3 \rho \kappa} 
    = \frac{L_*  - L_{\rm{conv}}}{4\pi r^2 dT/dr},
\end{equation}
where we use the second definition because it produces a smoother profile.
We define the heating function 
\begin{equation}
    \begin{split}
        \mathcal{H} &= \rho \epsilon - \frac{1}{4\pi r^2}\frac{d}{dr}\left(-4\pi r^2 k_{\rm{rad}}\frac{dT}{dr}\right)\\
        &=\frac{1}{4\pi r^2}\left[\frac{d L_*}{dr} - \frac{d}{dr}\left(-4 \pi r^2 k_{\rm{rad}}\frac{dT}{dr}\right)\right]
    \end{split}
\end{equation}
We find that the second equality produces a smoother profile than the first quantity, but use the first quantity (with $\rho \epsilon$) for the innermost radial points where the numerical derivative of $L_*$ is less well-defined.
We nondimensionalize using $\Lnd = 8.21 \times 10^{10}$ cm, $\Tnd = 3.63 \times 10^7$ K, $\mnd = 4.30 \times 10^{33}$ g, and $\tnd = 4.66 \times 10^{5}$ s.

Given this nondimensionalization and these MESA profiles, we control the turbulence and resolution requirement of a simulation by specifying a nondimensional viscosity $\nu_{\rm{nd}}$
We then compute the following nondimensional radial profiles:
\begin{enumerate}
    \item $\chi_{\rm{nd}} = (k_{\rm{rad}}/[\rho c_P])(\tnd/\Lnd^2)$. When $\chi_{\rm{nd}} < \nu_{\rm{nd}}$, we set $\chi_{\rm{nd}} = \nu_{\rm{nd}}$.
    \item $\ln\rho$ is set to be $\ln[\rho (\Lnd^3/\mnd)]$.
    \item $\ln T$ is set to be $\ln(T / \Tnd)$.
    \item $T$ is set to be $T/\Tnd$.
    \item $\grad T$ is set to be $(d\ln T/dr)(\Lnd/\Tnd)$.
    \item $\tilde{\mathcal{H}} = (H/[\rho T]) (\Tnd\tnd^3/\Lnd^2)$.
    \item $\justgrad s_0 = (ds/dr) (\Tnd\tnd^2/\Lnd)$
\end{enumerate}

Some fields with particularly sharp transitions (e.g., $\grad s_0$, $\mathcal{H}$) require additional smoothing.
In Fig.\todo{put fig in}, we show comparisons between the stratification felt by the Dedalus simulation and the original MESA model.

